% =====================================================================
% ====                                                             ====
% ==== Synopsis template, revised from diploma thesis short paper
% ====                                                             ====
% ==== Henrik B�rbak Christensen Winter 2004-2005
% ====                                                             ====
% =====================================================================


\documentclass[a4paper]{article}

\usepackage[latin1]{inputenc}
\usepackage{palatino}
\usepackage{color}
\usepackage{hyperref}
\usepackage[german]{babel}
\usepackage{tikz}

\usetikzlibrary{calc,trees,positioning,arrows,chains,shapes.geometric,%
	decorations.pathreplacing,decorations.pathmorphing,shapes,%
	matrix,shapes.symbols}

\tikzset{
	>=stealth',
	punktchain/.style={
		rectangle, 
		rounded corners, 
		% fill=black!10,
		draw=black, very thick,
		text width=10em, 
		minimum height=3em, 
		text centered, 
		on chain},
	line/.style={draw, thick, <-},
	element/.style={
		tape,
		top color=white,
		bottom color=blue!50!black!60!,
		minimum width=8em,
		draw=blue!40!black!90, very thick,
		text width=10em, 
		minimum height=3.5em, 
		text centered, 
		on chain},
	every join/.style={->, thick,shorten >=1pt},
	decoration={brace},
	tuborg/.style={decorate},
	tubnode/.style={midway, right=2pt},
}

\hypersetup{
	colorlinks=true,
	linkcolor=black,
	filecolor=magenta,
	urlcolor=cyan
}


%% a point to check
\definecolor{checkcolor}{rgb}{0.75, 0.75, 0.75}
\newsavebox{\definitionbox}
\newenvironment{checkit}{%
\begin{lrbox}{\definitionbox}
\begin{minipage}[t]{0.85\textwidth}%
}%
{\end{minipage}\end{lrbox}%
\begin{center}{\colorbox{checkcolor}{\usebox{\definitionbox}}}%
\end{center}}


\title{Einf�hrung in die Wirtschaftsinformatik Zusammenfassung}

\author{K.M et S.F}

\date{2017}


\begin{document}

\maketitle

\sloppy
\begin{checkit}
Folgende Zusammenfassung ersetzt weder die Vorlesung noch �bungen, sondern dient als Lernhilfe zur Vorbereitung f�r die Klausur (Einf�hrung in die Wirtschaftsinformatik). Es gibt keine Garantie auf Vollst�ndigkeit und Richtigkeit dieses Dokuments.
\end{checkit}

\section*{Lerneinheit 1}
\subsection*{Lernziele}
\begin{enumerate}
	%%there has to be a better to solve this italics thing
	\item \hyperref[le1-1]{\textit{Sie erkennen die Bedeutung der Wirtschaftsinformatik.}}
	\item \hyperref[le1-2]{\textit{Sie kennen die Paradigmen und Teilgebiete der Wirtschaftsinformatik.}}
	\item \hyperref[le1-3]{\textit{Sie erhalten Einblick in das Berufsfeld Wirtschaftsinformatik.}}
	\item \hyperref[le1-4]{\textit{Sie kennen die Bedeutung der Ressource Information und von Informationssystemen f�r Unternehmen.}}
	\item \hyperref[le1-5]{\textit{Sie kennen die Bestandteile und Eigenschaften von Informationssystemen.}}
	\item \hyperref[le1-6]{\textit{Sie k�nnen die verschiedenen Arten von Informationssystemen systematisieren.}}
\end{enumerate}

\subsection*{Frage 1}
\label{le1-1}
\begin{itemize}
	\item \textbf{Gegenstand} der Wirtschaftsinformatik sind Informations- und Kommunikationssysteme in Wirtschaft und Verwaltung.
	\item \textbf{Ziel} ist die optimale Bereitstellung von Information und Kommunikation nach wirtschaftlichen Kriterien
\end{itemize}

\subsection*{Frage 2}
\label{le1-2}
\begin{itemize}
	\item \textbf{Modellierung} $\rightarrow$ Reduzierung der Komplexit�t
	\item \textbf{Integrationswissenschaft} $\rightarrow$ Zusammenh�nge Menschen-Organisation-IKT
	\item \textbf{Gestaltung} $\rightarrow$ betrieblicher Informationssysteme
\end{itemize}

\subsection*{Frage 3}
\label{le1-3}
\begin{enumerate}
	\item Hardware die f�r \textbf{Systeme} ben�tigt werden
	\item Kosten-Nutzen-Analyse f�r das geplante \textbf{System} 
	\item \textbf{System}einf�hrung gestalteten
	\item Entscheidung zwischen Standardsoftware und speziell entwickelte Software
\end{enumerate}

\subsection*{Frage 4}
\label{le1-4}
%% SOURCE OF THIS SORCERY I JUST DO THE EDIT http://www.texample.net/tikz/examples/assignment-structure/
\subsubsection*{Zeichen, Daten, Information und Wissen}
\begin{center}
\begin{tikzpicture}
[node distance=.8cm,
start chain=going above,]
\node[punktchain, draw=none](woop){}; %%nice little hack
\node (asym) [punktchain ]  {Zeichen};
\node[punktchain, join,] (disk) {Daten};
\node[punktchain, join,] (makro) {Information};
\node[punktchain, join] (konk) {Wissen};


%% No. 2
\draw[tuborg, decoration={brace}] let \p1=(disk.north), \p2=(makro.south) in	
($(2, \y1)$) -- ($(2, \y2)$) node[tubnode] {KONTEXT};

\draw[tuborg, decoration={brace}] let \p1=(asym.north), \p2=(disk.south) in ($(2, \y1)$) -- ($(2, \y2)$) node[tubnode] {SYNTAX};

\draw[tuborg, decoration={brace}] let \p1=(makro.north), \p2=(konk.south) in ($(2, \y1)$) -- ($(2, \y2)$) node[tubnode] {VERNETZUNG};

\draw[tuborg, decoration={brace}] let \p1=(woop.north), \p2=(asym.south) in ($(2, \y1)$) -- ($(2, \y2)$) node[tubnode] {ZEICHENVORRAT};

\end{tikzpicture}
\end{center}
\hrulefill
\subsubsection*{Informationssystem}
%\textbf{Informationssystem}:
\begin{quote}
Es handelt sich um soziotechnische Systeme, die menschliche und maschinelle Komponenten als Aufgabentr�ger umfassen, die voneinander abh�ngig sind, ineinandergreifen und oder zusammenwirken.
\end{quote}
\hrulefill
\subsubsection*{Information als Wirtschaftsgut}
%\hrulefill \\
Um Information als Wirtschaftsgut anzusehen muss eine relative Knappheit bestehen und �konomisch auf eine Nachfrage sto�en.
\\
\hrulefill

%\pagebreak

%\subsubsection*{}
\hrulefill
\subsubsection*{Informationslogistische Grundprinzip \texttt{MIEZO}}
%\textbf{Informationslogistische Grundprinzip} (\texttt{MIEZO}) 
\begin{itemize}
	\item in der richtigen \textbf{Menge}
	\item der richtigen \textbf{Information}
	\item \textbf{Ziel} ist die Bereitstellung/Vorhandensein
	\item in der \textbf{erforderlichen Qualit�t} 
	\item zum richtigen \textbf{Zeitpunkt} 
	\item am richtigen \textbf{Ort}
\end{itemize}

\subsection*{Frage 5}
\subsubsection*{Systeme}
%\textbf{Systeme} \\
Ein System ist eine Menge von Elementen miteinander in Beziehung stehen. Sie unterscheiden sich in \\
\begin{itemize}
	\item offen - geschlossen
	\item dynamisch - statisch
	\item komplex - einfach
\end{itemize}

\hrulefill

\subsubsection*{Charakteristika / Eigenschaften}

\begin{itemize}
	\item besteht aus \textbf{Eigenschaften} und/oder \textbf{Menschen}
	\item die Informationen \textbf{erzeugen} und/oder benutzen
	\item und die durch Kommunikationsbeziehungen miteinander verbunden sind
\end{itemize}


Describe the problem that your project work is focusing on. Try hard
to come up with a 5-10 line hypothesis that you are capable of confirm
or falsify. It is no problem to state a problem in half a page, and
no-one will notice that it is ill-defined (not even yourself!)
However, forcing yourself to write it in five lines only, requires lot
of precision that will force you to define your project much more
accurately.

You may also express it somewhat broader as a problem statement (still
short!), but ensure that it is in a form where you can
argue/demonstrate that you have analyzed and evaluated the problem.

A fine ``problem'' in a teaching context is also to apply theory in
practice and learn about its advantages and short-comings.

An important part of this section is also assumptions and
delimitations. What assumptions do you have about the project, the
process or the environment? What subproblems do you not address or
will not address? Almost any project is capable of consuming an
infinite amount of work and you do not have unlimited time and
resources as hand---therefore it is important to explicitly delimit
the problem and state what you intend to look into and what not.

A final thing is characterization: ensure that all the concepts you
use is well-defined or else give a definition. We know what 'object'
means in the usual sense, but a term like 'component' still has many
different meanings! Which one do you use? Also if you use company
specific concepts, be sure to define them as precisely as possible.

Use typography to make definitions and problem statements stand out in
the text. It is more difficult to overview a text where important
points are written inside large chunks of less import text than it is
to find them as

\begin{checkit}
Use typography to make important points like definitions, results,
problem statements, and other text that needs to be consulted often,
stand out in the text.
\end{checkit}

\section{Method}

How do you intend to analyse the problem? What theory, books, and
papers have you read that help you to analyse, discuss, and work with
the problem?


A short summary of how you are going to confirm/falsify the
hypothesis: what prototypes do you expect to build, how will you
evaluate and measure them, what techniques and tools are you going to
use, which people will you interview, how will you document processes
and products, how will you record you progress, how will you analyze
your work, how will each phase contribute to validating the
hypothesis?

\section{(Expected) Analyses and Results}

This the main body of your report where you outline what you have
done, how it contributes to analysing the problem, the results of your
work, argue why your results are correct, relate them to theory, and
document what has been achieved.

Remember that designs are often best described and supported by
diagrams and central algorithms are best expressed in small fragments
of real or pseudo code. Text like ``the server calls the 'update'
method which next calls the 'IAmBored' method in the \ldots'' are
terrible and next to impossible to read.

Avoid narrative writing styles like ``then we did X but it did not work, so
we tried Y and it worked better''. Rather, use a (problem, analysis,
solution) format: ``Problem: (describe short and precisely), Analysis:
(describe a set of problem solutions), Solution: (describe and argue
why a given solution was chosen).''

Remember that the primary objective with your report is to demonstrate
that you master the theory introduced in the course and your
analytical skills to ``think clever thoughts'' and related and discuss
your work. It is not to program a polished product ready for shipment.


\section{Related work}

[This section may be put in front of the hypothesis section or
  integrated into the method section, if it make the flow of text more
  natural.]

In this section you outline what literature and other work your
project build upon: papers, books, links to webpages, tutorials,
etc. All references should be resolved in the reference section, that
is do not use footnotes, or put the reference directly in the
text. For an example, look how references are cited in Bardram et
al.\cite{bardram}.

It is good to address how your work extends, use, or build upon the
cited work. 

\section{Conclusion}

Your synopsis should clearly state: abstract, motivation, hypothesis,
method, and (expected) analyses and results.


% ============================================================================
% === REFERENCES =============================================================
% ============================================================================

\bibliographystyle{abbrv}

\bibliography{bib}



\end{document}
\grid
