% =====================================================================
% ====                                                             ====
% ==== Synopsis template, revised from diploma thesis short paper
% ====                                                             ====
% ==== Henrik Bærbak Christensen Winter 2004-2005
% ====                                                             ====
% =====================================================================


\documentclass[a4paper]{article}

\usepackage[latin1]{inputenc}
\usepackage{palatino}
\usepackage{color}
\usepackage{hyperref}
\usepackage[german]{babel}
\usepackage{tikz}
\usepackage{adjustbox}
\usepackage{booktabs}
\newcommand{\tabitem}{~~\llap{\textbullet}~~}

\usetikzlibrary{calc,trees,positioning,arrows,chains,shapes.geometric,%
	decorations.pathreplacing,decorations.pathmorphing,shapes,%
	matrix,shapes.symbols}

\newenvironment*{dummyenv}{}{}


\tikzset{
	>=stealth',
	punktchain/.style={
		rectangle, 
		rounded corners, 
		% fill=black!10,
		draw=black, very thick,
		text width=10em, 
		minimum height=3em, 
		text centered, 
		on chain},
	line/.style={draw, thick, <-},
	element/.style={
		tape,
		top color=white,
		bottom color=blue!50!black!60!,
		minimum width=8em,
		draw=blue!40!black!90, very thick,
		text width=10em, 
		minimum height=3.5em, 
		text centered, 
		on chain},
	every join/.style={->, thick,shorten >=1pt},
	decoration={brace},
	tuborg/.style={decorate},
	tubnode/.style={midway, right=2pt},
}

\hypersetup{
	colorlinks=true,
	linkcolor=black,
	filecolor=magenta,
	urlcolor=cyan
}

%% a point to check
\definecolor{checkcolor}{rgb}{0.75, 0.75, 0.75}
\newsavebox{\definitionbox}
\newenvironment{checkit}{%
\begin{lrbox}{\definitionbox}
\begin{minipage}[t]{0.85\textwidth}%
}%
{\end{minipage}\end{lrbox}%
\begin{center}{\colorbox{checkcolor}{\usebox{\definitionbox}}}%
\end{center}}


\title{Einführung in die Wirtschaftsinformatik Zusammenfassung}

\author{K.M et S.F}

\date{2017}


\begin{document}

\maketitle

\sloppy
\begin{checkit}
Folgende Zusammenfassung ersetzt weder die Vorlesung noch Übungen, sondern dient als Lernhilfe zur Vorbereitung für die Klausur (Einführung in die Wirtschaftsinformatik). Es gibt keine Garantie auf Vollständigkeit und Richtigkeit dieses Dokuments.
\end{checkit}

\section*{Lerneinheit 1}
\subsection*{Lernziele}
\begin{enumerate}
	%%there has to be a better to solve this italics thing
	\item \hyperref[le1-1]{\textit{Sie erkennen die Bedeutung der Wirtschaftsinformatik.}}
	\item \hyperref[le1-2]{\textit{Sie kennen die Paradigmen und Teilgebiete der Wirtschaftsinformatik.}}
	\item \hyperref[le1-3]{\textit{Sie erhalten Einblick in das Berufsfeld Wirtschaftsinformatik.}}
	\item \hyperref[le1-4]{\textit{Sie kennen die Bedeutung der Ressource Information und von Informationssystemen für Unternehmen.}}
	\item \hyperref[le1-5]{\textit{Sie kennen die Bestandteile und Eigenschaften von Informationssystemen.}}
	\item \hyperref[le1-6]{\textit{Sie können die verschiedenen Arten von Informationssystemen systematisieren.}}
\end{enumerate}

\subsection*{Frage 1}
\label{le1-1}
\begin{itemize}
	\item \textbf{Gegenstand} der Wirtschaftsinformatik sind Informations- und Kommunikationssysteme in Wirtschaft und Verwaltung.
	\item \textbf{Ziel} ist die optimale Bereitstellung von Information und Kommunikation nach wirtschaftlichen Kriterien
\end{itemize}

\subsection*{Frage 2}
\label{le1-2}
\begin{itemize}
	\item \textbf{Modellierung} $\rightarrow$ Reduzierung der Komplexität
	\item \textbf{Integrationswissenschaft} $\rightarrow$ Zusammenhänge Menschen-Organisation-IKT
	\item \textbf{Gestaltung} $\rightarrow$ betrieblicher Informationssysteme
\end{itemize}

\subsection*{Frage 3}
\label{le1-3}
\begin{enumerate}
	\item Hardware die für \textbf{Systeme} benötigt werden
	\item Kosten-Nutzen-Analyse für das geplante \textbf{System} 
	\item \textbf{System}einführung gestalteten
	\item Entscheidung zwischen Standardsoftware und speziell entwickelte Software
\end{enumerate}

\subsection*{Frage 4}
\label{le1-4}
%% SOURCE OF THIS SORCERY I JUST DO THE EDIT http://www.texample.net/tikz/examples/assignment-structure/
\subsubsection*{Zeichen, Daten, Information und Wissen}
\begin{center}
\begin{tikzpicture}
[node distance=.8cm,
start chain=going above,]
\node[punktchain, draw=none](woop){}; %%nice little hack
\node (asym) [punktchain ]  {Zeichen};
\node[punktchain, join,] (disk) {Daten};
\node[punktchain, join,] (makro) {Information};
\node[punktchain, join] (konk) {Wissen};


%% No. 2
\draw[tuborg, decoration={brace}] let \p1=(disk.north), \p2=(makro.south) in	
($(2, \y1)$) -- ($(2, \y2)$) node[tubnode] {KONTEXT};

\draw[tuborg, decoration={brace}] let \p1=(asym.north), \p2=(disk.south) in ($(2, \y1)$) -- ($(2, \y2)$) node[tubnode] {SYNTAX};

\draw[tuborg, decoration={brace}] let \p1=(makro.north), \p2=(konk.south) in ($(2, \y1)$) -- ($(2, \y2)$) node[tubnode] {VERNETZUNG};

\draw[tuborg, decoration={brace}] let \p1=(woop.north), \p2=(asym.south) in ($(2, \y1)$) -- ($(2, \y2)$) node[tubnode] {ZEICHENVORRAT};

\end{tikzpicture}
\end{center}
\hrulefill
\subsubsection*{Informationssystem}
%\textbf{Informationssystem}:
\begin{quote}
Es handelt sich um soziotechnische Systeme, die menschliche und maschinelle Komponenten als Aufgabenträger umfassen, die voneinander abhängig sind, ineinandergreifen und oder zusammenwirken.
\end{quote}
\hrulefill
\subsubsection*{Information als Wirtschaftsgut}
%\hrulefill \\
Um Information als Wirtschaftsgut anzusehen muss eine relative Knappheit bestehen und ökonomisch auf eine Nachfrage stoßen.
\hrulefill

%\pagebreak

%\subsubsection*{}
\hrulefill
\subsubsection*{Informationslogistische Grundprinzip \texttt{MIEZO}}
%\textbf{Informationslogistische Grundprinzip} (\texttt{MIEZO}) 
\begin{itemize}
	\item in der richtigen \textbf{Menge}
	\item der richtigen \textbf{Information}
	\item \textbf{Ziel} ist die Bereitstellung/Vorhandensein
	\item in der \textbf{erforderlichen Qualität} 
	\item zum richtigen \textbf{Zeitpunkt} 
	\item am richtigen \textbf{Ort}
\end{itemize}

\subsection*{Frage 5}
\label{le1-5}
\subsubsection*{Systeme}
%\textbf{Systeme} \\
Ein System ist eine Menge von Elementen miteinander in Beziehung stehen. Sie unterscheiden sich in
\begin{itemize}
	\item offen -- geschlossen
	\item dynamisch -- statisch
	\item komplex -- einfach
\end{itemize}
\hrulefill

\subsubsection*{Charakteristika / Eigenschaften}
\begin{itemize}
	\item besteht aus \textbf{Eigenschaften} und/oder \textbf{Menschen}
	\item die Informationen \textbf{erzeugen} und/oder benutzen
	\item und die durch Kommunikationsbeziehungen miteinander verbunden sind
\end{itemize}
\hrulefill
\subsubsection*{Grundfragen bei der Gestaltung von Informationssystem}
\begin{itemize}
	\item \textbf{Wozu} wird die Information gebraucht \textit{(Auswertungszweck)}
	\item \textbf{Wer} (Sender) soll \textbf{wen} (Empfänger) über \textbf{was} (Inhalt, Genauigkeit) informieren?
	\item \textbf{Wann} (Termine) soll informiert werden?
	\item \textbf{Wie} (Art, Form, Methode, Weg) soll informiert werden?
\end{itemize}

\subsection*{Frage 6}
\label{le1-6}
\subsubsection*{Arten von Informationssytemen:}
\begin{itemize}
	\item \textbf{betriebliches IS:} unterstützt Leistungsprozesse und Austauschbeziehungen innerhalb des Betriebs sowie zwischen Betrieb und Umwelt
	\item \textbf{rechnergestütztes IS:} basiert und Einsatz von Informationstechnik (Mensch--Maschine System)
	\item \textbf{integrierte IS:} Daten, Funktionen und Verfahren sind über gemeinsame Strukturen verknüpft
	\item \textbf{nicht integrierte IS:} Schnittstellen verknüpfen die Systeme miteinander
\end{itemize}

\pagebreak

\section*{Lerneinheit 2}
\subsection*{Lernziele}
\begin{enumerate}
	%%there has to be a better to solve this italics thing
	\item \hyperref[le2-1]{\textit{Sie kennen die wichtigsten technischen Entwicklungslinien Kapazitätssteigerung, Social Media, Mobility \& Consumerization, Analytics/Big Data, Cloud Computing.}}
	\item \hyperref[le2-2]{\textit{Sie wissen, was unter dem Trend der Digitalisierung verstanden wird und können die Auswirkungen abschätzen.}}
	\item \hyperref[le2-3]{\textit{Sie lernen, dass Informationsverarbeitung kein Selbstzweck ist, sondern zum Erreichen der Unternehmensziele dient.}}
	\item \hyperref[le2-4]{\textit{Sie wissen, dass es einen Zusammenhang zwischen Unternehmensstrategie und -umsetzung, Organisation und IKT gibt.}}
	\item \hyperref[le2-5]{\textit{Sie kennen die Prozessorientierung als ein wesentliches organisatorisches Paradigma.}}
\end{enumerate}

\subsection*{Frage 1}
\label{le2-1}
\subsubsection*{Technologie-Trends -- Gartner's Hype Cycle}
\begin{itemize}
	\item Beschreibung technischer Trends
	\item Innovatoren $\rightarrow$ Frühe Anwender $\rightarrow$ Frühe Mehrheit $\rightarrow$ Späte Mehrheit $\rightarrow$ Nachzügler
\end{itemize}
\hrulefill
\subsubsection*{Moore's Law}
\begin{itemize}
	\item Zeitraum: 12--18 Monate
	\item Verdopplung der Transistoren
	\item Halbierung der Grundfläche
	\item Kostenverringerung um 30--50\%
\end{itemize}
\hrulefill
\subsubsection*{Web 2.0}
Das für eine Reihe \textbf{interaktiver} und \textbf{kollaborativer} Elemente des Internets, speziell des World Wide Web, verwendet wird. Dabei konsumiert der Nutzer nicht nur den Inhalt, er stellt als Prosument \textbf{selbst Inhalt zur Verfügung}.
%\appendix
\part*{}
\hrulefill
\subsubsection*{Mobility \& Consumerization}
Anbieter legen ihren Fokus in IT Produkt und Service Bereich auf die Kunden (hohe \textbf{Bedienfreundlichkeit}, neue \textbf{Anwendungsmöglichkeiten})
\part*{}
\hrulefill

\subsubsection*{Planbarkeit}
\begin{itemize}
	\item Responsiveness (\textit{agility})
	\item Resilience (\textit{robustness}) 
	\item Readiness (\textit{anticipation}) 
	\item Recursion (\textit{experimentation})
\end{itemize}
\hrulefill

\subsubsection*{Big Data \& Datenverfügbarkeit}
\begin{itemize}
	\item 2.5 Exabytes pro Jahr -- verdoppelt sich alle 40 Monate
	\item Geschwindigkeit -- real-time
	\item Vielfalt der Daten -- GPS, Bilder, Nachrichten
\end{itemize}
\hrulefill

\subsubsection*{Cloud Computing}
Beschreibt die Bereitstellung von IT-Infrastuktur und IT-Leistungen im Internet.
%\begin{dummyenv}
%\end{dummyenv}
\part*{}
\hrulefill

\subsection*{Frage 2}
\label{le2-2}
\subsubsection*{Digitalisierung}
...ist die Transformation analoger Werte in digigtaler Form. Mit der Absicht sie zu speichern und/oder verarbeiten.

\subsubsection*{Plattform}
...beschreibt eine einheitliche Grundlage, auf der Anwendungssoftware ausgeführt und entwickelt werden können.

\subsubsection*{Plattformökosystem}
...Plattform und alle Stakeholder die auf ihr interagieren. \\
\hrulefill

\subsubsection*{Zwei Perspektiven auf Plattform Ökosysteme}
%\begin{adjustbox}{max width=\textwidth}
%	\begin{tabular}{p{.7\textwidth}}{|r|c|c}
%		\hline
%		\textbf{Plattform} & \textbf{Technologie-orientiert} & \textbf{Markt-orientiert} \\
%		\hline
%		\textbf{Zweck} & Mitgestaltung der Wertschöpfung, Innovation & Abgleich von Angebot und Nachfrage, Informationsaustausch  \\
%		\hline
%		\textbf{Anwendung} & Software -- Hardware & Marktplatz -- Community \\
%		\hline
%	\end{tabular}
%\end{adjustbox}

%WILL FIX LATER
\begin{adjustbox}{max width=\linewidth}
	\begin{tabular}{|r|c|c|}
		\hline
		\textbf{Plattform} & \textbf{Technologie-orientiert} & \textbf{Markt-orientiert} \\
		\hline
		\textbf{Zweck} & Mitgestaltung der Wertschöpfung, Innovation & Abgleich von Angebot und Nachfrage, Informationsaustausch  \\
		\hline
		\textbf{Anwendung} & Software -- Hardware & Marktplatz -- Community \\
		\hline
	\end{tabular}
\end{adjustbox}

\subsection*{Frage 3}
\label{le2-3}
\subsubsection*{Technik, Innovationen, Wettbewerb}
MISSING SELF-MADE DIAGRAM.

\subsection*{Frage 4}
\label{le2-4}
\subsubsection*{Veränderung der Rolle der IKT}
\begin{itemize}
	\item [] \textbf{1. Phase:} Massenverarbeitung (50er-60er Jahre)
	\item [] \textbf{2. Phase:} Produktivitätssteigerung (70er-80erJahre)
	\item [] \textbf{3. Phase:} Strategischer Einsatz von Informationssysteme
	\item [] \textbf{4. Phase:} Informationsbereitstellung (90er Jahre)
\end{itemize}
\hrulefill

\subsubsection*{Einfluss IKT auf Organisation:}
\begin{itemize}
	\item Überführung unstrukturierter Abläufe in routinemäßige Abläufe
	\item Beschleunigung wertschöpfender Aktivitäten
	\item Ersatz und Reduktion menschlicher Arbeit
	\item Verfolgung von Input, Output und Status
\end{itemize}


\subsection*{Frage 5}
\label{le2-5}
\subsubsection*{Prozessorientierung als organisatorisches Paradigma}
Organisation die auf IT gestutzte Prozessorientierung aufbaut.

\pagebreak %maybe a good idea to pagebreak between each LE

\section*{Lerneineit 3}
\subsection*{Lernziele}
\begin{enumerate}
	\item \hyperref[le3-1]{\textit{Sie kennen die Notwendigkeit der Verwendung von Modellen sowie verschiedene Arten von Modellen}}
	\item \hyperref[le3-2]{\textit{Sie verstehen den Unterschied zwischen Modell und Referenzmodell (Referenzbehauptung, Vor- und Nachteile)}}
	\item \hyperref[le3-3]{\textit{Sie haben einen Überblick über verschiedene Referenzmodelle (Handel, Industrie, Lieferketten)}}
\end{enumerate}

\subsection*{Frage 1}
\label{le3-1}
\subsubsection*{Warum Modelle?}
%
%\begin{adjustbox}{max width=\linewidth}
%\begin{table}
%\begin{tabular}{r c}
%	Grundzweck & Abbildung der Realität zur Reduktion von Komplexität \\
%	Fragen beim Modellieren & \tabitem Wovon
%\end{tabular}
%\end{table}
%\end{adjustbox}
%\\
\centering
	\begin{tabular}{ll}
		\toprule
			\multicolumn{2}{c}{\textbf{Grundzweck}: Reduktion von Komplexität} \\[.5\normalbaselineskip]
		\multicolumn{2}{c}{Modell ist stets Modell:} \\
		\midrule
		\tabitem Wovon? & Gegenstand \\
		\tabitem Wozu? & Zweck \\
		\tabitem Für wen? & Zielgruppe, Adressat \\
		\bottomrule
	\end{tabular}
\\[1\normalbaselineskip]

\hrulefill

\raggedright
\subsubsection*{Elemente von Modellen}
\begin{itemize}
	\item Abbildungsregeln
	\item Modellsubjekt
	\item Abzubildene Realität
	\item Adressaten der Modelbetrachtung
\end{itemize}
\hrulefill

\subsubsection*{Schritte der Modellierung}
%\begin{tikzpicture}
%
%\def \n {5}
%\def \radius {3cm}
%\def \margin {8} % margin in angles, depends on the radius
%
%\foreach \s in {1,...,\n}
%{
%	\node[draw, circle] at ({360/\n * (\s - 1)}:\radius) {$\s$};
%	\draw[->, >=latex] ({360/\n * (\s - 1)+\margin}:\radius) 
%	arc ({360/\n * (\s - 1)+\margin}:{360/\n * (\s)-\margin}:\radius);
%}
%\end{tikzpicture}

\begin{tikzpicture}[
start chain = going right,
block/.style = {rectangle, draw, rounded corners, 
	text width=6em, align=center, minimum height=4em,
	on chain},
every pin/.style = {inner sep=1mm, align=center, font=\footnotesize,
	pin distance=9mm, pin edge={angle 60-, solid, black}},
]
\node[block] (A) {Ausschnitt aus der Realität};
\node[block] (B) {Ist-Modell};
\node[block] (C) {Soll-Modell};
\node[block] (D) {Realität schaffen/ändern};

\linespread{0.9}
\draw[-latex'] (A) to node[inner sep=0pt,
pin=above:aktive (Re-)konstruktion\\
Modellierung] {}    (B);

\draw[-latex'] (B) to node[inner sep=0pt,
pin=above:Fantasie] {}    (C);

\draw[-latex'] (C) to node[inner sep=0pt,
pin=above:Konstruktion] {}    (D);
\end{tikzpicture}
\\ \hrulefill
\pagebreak

\subsubsection*{Arten von Modellen}
\begin{itemize}
	\item \textbf{deskriptiv} (Abbild): Erklärungs-, Prognosemodelle
	\item \textbf{transient} (Ab- und Vorbild)
	\item \textbf{präskriptiv}(Vorbild): Gestaltung-, Optimierungsmodelle
\end{itemize}

%
%\begin{tabular}{ll}
%	\multicolumn{2}{l}{\textbf{Arten von Modellen}} \\[.5\normalbaselineskip]
%	\tabitem \textbf{deskriptiv} (Abbild): & Erklärungs-, Prognosemodelle \\[.5\normalbaselineskip]
%	\tabitem \textbf{transient} (Ab- und Vorbild) & \\[.5\normalbaselineskip]
%%	\tabitem \multicolumn{2}{l}{transient Ab- und Vorbild} \\[.5\normalbaselineskip]
%	\tabitem \textbf{präskriptiv} (Vorbild): & Zielgruppe, Adressat \\[.5\normalbaselineskip]
%\end{tabular}
%\\[1\normalbaselineskip]

\hrulefill

\raggedright

\subsubsection*{Aspekte der Modellierung}
\begin{itemize}
	\item schaffen Transparenz über Elemente und Beziehungen im Unternehmen
	\item erklären Funktionsweise des Unternehmens
	\item erleichtert die Kommunikation im Unternehmen
\end{itemize}

\subsection*{Frage 2}
\label{le3-2}
\subsubsection*{Referenz:}
\begin{quote}
	auf etwas zurückführen, sich auf etwas beziehen, berichten.
\end{quote}
\subsubsection*{Ein Referenz-Informationsmodell ist ...}
\begin{itemize}
	\item das \textbf{immaterielle Abbild}
	\item der in einem \textbf{realen} oder \textbf{gedachten} betrieblichen \textbf{Objektsystem}
	\item verarbeiteten Informationen,
	\item das für \textbf{Zwecke des Informationssystem-} und Organisationsgestalters
	\item \textbf{Empfehlungscharakter} besitzt und
	\item als \textbf{Bezugspunkt} für \textbf{unternehmensspezifische Informationsmodelle} dienen kann.
\end{itemize}

\subsubsection*{Vereinfacht/tldr:}
\begin{quote}
	\texttt{Das Referenzmodell stellt somit ein Modellmuster dar, das als idealtypisches Modell für die Klasse der zu modellierenden Sachverhalte betrachtet werden kann.}
\end{quote}

\subsubsection*{Beschreibung:}
\begin{itemize}
	\item Normativer Charakter von Referenzmodellen (\textbf{Gestaltungsempfehlungen})
	\item \textbf{Heterogenität} der Referenzmodelle (z.B. branchenspezifische Datenmodelle, \hyperref{https://en.wikipedia.org/wiki/OSI_model}{}{}{ISO-OSI-Schichtenmodell})
\end{itemize}

\subsubsection*{Anforderungen}
\begin{itemize}
	\item \textbf{Allgemeingültig}keitsanspruch von Referenzmodellen
	\subitem Problem: Wahl eines adäquaten Abstraktionsgrades
	\item \textbf{Robustheit der Modelle} gegenüber \textbf{Änderungen der Real Welt}
	\subitem Flexibilität: Durchführung von Veränderungen mit \textbf{geringem Aufwand}
	\item \textbf{Konsistenzforderung} an Referenzmodelle
\end{itemize}

%\centering
\begin{adjustbox}{max width=\linewidth}
\begin{tabular}{l|l}
	\toprule
	\multicolumn{2}{c}{\textbf{Modellvergleich}} \\[.5\normalbaselineskip]
	%\multicolumn{2}{c}{Modell ist stets Modell:} \\
	Modell & Referenzmodell \\
	\midrule
	\tabitem eine Abbildung eines Systems von Objekten & \tabitem für Wiederverwendung empfohlenes Modell \\
	\tabitem bestimmten Zweck & \tabitem Modell das für Konstruktion weiterer Modelle genutzt wird \\
	\tabitem konstruiert Abbildung realer Objekte für bestimmte Adressaten & \tabitem Konstruktion semantischer Gemeinsamkeiten in den Modellen \\
	\bottomrule
\end{tabular}
\end{adjustbox}
\\[1\normalbaselineskip]

\subsubsection*{Vor- und Nachteile}
\begin{itemize}
	\item Spezialisierung, nicht individuell angepasst
	\item Kostenersparnis durch Nutzung vom Referenzmodell (das Rad nicht neu erfinden)
	\item leicht modifizierbar
	\item keine Innovation durch Referenzmodelle
\end{itemize}
\hrulefill

\subsubsection*{Metamodell}
\begin{itemize}
	\item Abbildung von Modell \& Modellbildung als Gegenstand der Modellierung mit Fokus auf \textbf{Syntax des Modellsystems}
	\item Prozess- oder Sprachenfokus
	\item abstrahiert von der Semantik des Modells
\end{itemize}

\subsubsection*{Ordnungsrahmen}
\begin{quote}
	Schafft aggregierten Überblick über wesentliche Funktionsbereichen einer Domäne.
\end{quote}
%\hrulefill

\pagebreak

\subsection*{Frage 3}
\label{le3-3}
\subsubsection*{Handels-H}
\begin{itemize}
	\item für Handelsunternehmen 
	\item betriebw. Kernfunktionalitäten 
	\item Beschaffen, Lagern, Verkaufen; Betriebw. administr. Aufgaben; dispositive Aufgaben
\end{itemize}

\subsubsection*{CIM (Computer Integrated Manufacturing Model)}
\begin{itemize}
	\item CIM beschreibt den integrierten EDV-Einsatz in allen mit der Produktion zusammenhängenden Betriebsbereichen.
	\item CIM umfasst das informationstechnologische Zusammenwirken zwischen CAD, CAP, CAM, CAQ und PPS.
	\item Bedingung: gemeinsame, bereichsübergreifende Nutzung der
	\item Datenbasis
\end{itemize}

%INSERT IMAGE/REFERENCE HERE INSTEAD
\subsubsection*{SCOR-Modell (Supply Chain Operations Reference Model)}
\begin{itemize}
	\item 
\end{itemize}

\section*{Lerneinheit 4}
\subsection*{Lernziele}
\begin{enumerate}
	\item \hyperref[le4-1]{\textit{Sie lernen ARIS (Architektur Integrierter Informationssysteme) kennen und können die Verknüpfung der Daten- und Funktionssicht mit Hilfe von EPKs modellieren.}}
	\item \hyperref[le4-2]{\textit{Sie sind in der Lage, einfache betriebswirtschaftliche Sachverhalte und Geschäftsprozesse in Datenmodelle zu überführen.}}
	\item \hyperref[le4-3]{\textit{Sie verstehen das Paradigma der Objektorientierung, kennen die Unified Modeling Language (UML) und können einfache Klassendiagramme und Anwendungsfalldiagramme modellieren.}}
\end{enumerate}

\subsection*{Frage 1}
\subsubsection*{ARIS (Haus)}
\begin{quote}
	allgemeiner Bezugsrahmen für Geschäftsprozessmodellierung.
\end{quote}

%\begin{adjustbox}{max width=\linewidth}
	\begin{tabular}{l|l|l}
		\toprule
		\multicolumn{3}{c}{\textbf{Sichten}} \\[.5\normalbaselineskip]
		%\multicolumn{2}{c}{Modell ist stets Modell:} \\
		\textbf{Sicht} & \textbf{Inhalt} & \textbf{Beispiel} \\
		\midrule
		\textbf{Datensicht} & beschreibt Informationsobjekte zu Repräsentation von Ereignissen und Zuständen. Auftrag ist abgewickelt & Auftrag ist abgewickelt \\[0.5\normalbaselineskip]
		\textbf{Funktionssicht} & beschreibt Funktionen und ihre Zusammenhänge in Form von Funktionsbäumen & (Teil)funktionen der Auftragsabwicklung \\[0.5\normalbaselineskip]
		\textbf{Organisationssicht} & beschreibt Struktur und Beziehungen und Aufgabenträgern und Organisationseinheiten & Herr M. Abt. Auftragsannahme \\[0.5\normalbaselineskip]
		\textbf{Steuerungssicht} & beschreibt die Verbindung zwischen den Sichten & Prozesskette: Auftragsabwicklung \\[0.5\normalbaselineskip]
		\bottomrule
	\end{tabular}
%\end{adjustbox}
\\[1\normalbaselineskip]

\begin{tabular}{|p{0.8\linewidth}|p{0.6\linewidth} |}
	TEXT TEXT TEXT & TEXT TEXT \\
\end{tabular}

%\raggedright

Describe the problem that your project work is focusing on. Try hard
to come up with a 5-10 line hypothesis that you are capable of confirm
or falsify. It is no problem to state a problem in half a page, and
no-one will notice that it is ill-defined (not even yourself!)
However, forcing yourself to write it in five lines only, requires lot
of precision that will force you to define your project much more
accurately.

You may also express it somewhat broader as a problem statement (still
short!), but ensure that it is in a form where you can
argue/demonstrate that you have analyzed and evaluated the problem.

A fine ``problem'' in a teaching context is also to apply theory in
practice and learn about its advantages and short-comings.

An important part of this section is also assumptions and
delimitations. What assumptions do you have about the project, the
process or the environment? What subproblems do you not address or
will not address? Almost any project is capable of consuming an
infinite amount of work and you do not have unlimited time and
resources as hand---therefore it is important to explicitly delimit
the problem and state what you intend to look into and what not.

A final thing is characterization: ensure that all the concepts you
use is well-defined or else give a definition. We know what 'object'
means in the usual sense, but a term like 'component' still has many
different meanings! Which one do you use? Also if you use company
specific concepts, be sure to define them as precisely as possible.

Use typography to make definitions and problem statements stand out in
the text. It is more difficult to overview a text where important
points are written inside large chunks of less import text than it is
to find them as

\begin{checkit}
Use typography to make important points like definitions, results,
problem statements, and other text that needs to be consulted often,
stand out in the text.
\end{checkit}

\section{Method}

How do you intend to analyse the problem? What theory, books, and
papers have you read that help you to analyse, discuss, and work with
the problem?


A short summary of how you are going to confirm/falsify the
hypothesis: what prototypes do you expect to build, how will you
evaluate and measure them, what techniques and tools are you going to
use, which people will you interview, how will you document processes
and products, how will you record you progress, how will you analyze
your work, how will each phase contribute to validating the
hypothesis?

\section{(Expected) Analyses and Results}

This the main body of your report where you outline what you have
done, how it contributes to analysing the problem, the results of your
work, argue why your results are correct, relate them to theory, and
document what has been achieved.

Remember that designs are often best described and supported by
diagrams and central algorithms are best expressed in small fragments
of real or pseudo code. Text like ``the server calls the 'update'
method which next calls the 'IAmBored' method in the \ldots'' are
terrible and next to impossible to read.

Avoid narrative writing styles like ``then we did X but it did not work, so
we tried Y and it worked better''. Rather, use a (problem, analysis,
solution) format: ``Problem: (describe short and precisely), Analysis:
(describe a set of problem solutions), Solution: (describe and argue
why a given solution was chosen).''

Remember that the primary objective with your report is to demonstrate
that you master the theory introduced in the course and your
analytical skills to ``think clever thoughts'' and related and discuss
your work. It is not to program a polished product ready for shipment.


\section{Related work}

[This section may be put in front of the hypothesis section or
  integrated into the method section, if it make the flow of text more
  natural.]

In this section you outline what literature and other work your
project build upon: papers, books, links to webpages, tutorials,
etc. All references should be resolved in the reference section, that
is do not use footnotes, or put the reference directly in the
text. For an example, look how references are cited in Bardram et
al.\cite{bardram}.

It is good to address how your work extends, use, or build upon the
cited work. 

\section{Conclusion}

Your synopsis should clearly state: abstract, motivation, hypothesis,
method, and (expected) analyses and results.


% ============================================================================
% === REFERENCES =============================================================
% ============================================================================

\bibliographystyle{abbrv}

\bibliography{bib}



\end{document}
\grid
